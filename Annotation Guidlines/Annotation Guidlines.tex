\documentclass{article}
\usepackage{enumitem}
\usepackage{color}
\usepackage{forest}
\usepackage{tikz-qtree}
\usepackage{booktabs}
\usepackage{changepage}
\usepackage{lipsum}
\usepackage[fleqn]{amsmath}
\usepackage{fancyhdr}
\usepackage{tabularx,ragged2e,booktabs,caption}
\newcolumntype{C}[1]{>{\Centering}m{#1}}
\renewcommand\tabularxcolumn[1]{C{#1}}
\usepackage[a4paper, portrait, margin=1in]{geometry}
\linespread{1.5}
\title{Yelp Reviews Annotation Guidelines -- Draft}
\author{Qishen ``Justin'' Su, Kelley Lynch, Yuanyuan Ma}
\begin{document}

\newpage
\pagenumbering{arabic}
\maketitle

\tableofcontents

\newpage
\section{Introduction}
\paragraph{}
This document provides annotation guidelines for the Yelp restaurant reviews annotation task. The Yelp restaurant reviews annotation task involves (1) annotating the relation between a specific dish mentioned in a review and its described quality or characteristics, (2) annotating the relation between a dish and its ingredients or its parts, and (3) annotating the anaphors of dishes. These guidelines provide details on how to annotate Yelp restaurant reviews. 
\paragraph{}
This document is organized in the following manner: Section 2 is focused on annotating extent tags, i.e. FOOD, QUALITY and ANAPHORA, while Section 3 is focusing on annotating link tags, i.e. PART\_OF, OPINION, and COFERENCE. 
\paragraph{}
When examples of annotation are provided in this document, texts with different extent tags are marked in different colors: \textbf{\textcolor{blue}{blue}} is for FOOD, \textbf{\textcolor{red}{red}} is for QUALITY, and \textbf{\textcolor{yellow}{yellow}} is for ANAPHORA. A QUALITY tag contains three types: \textit{positive}, \textit{negative}, and \textit{neutral}, and in this document, a type of QUALITY will be presented in subscript. Then, each link tag is represented using a predicate argument structure given in Table 1. 
\\
\begin{minipage}{\linewidth}
\centering
\captionof{table}{Link Tag Predicate Argument} \label{tab:title} 
\begin{tabular}{ll}\toprule[1.5pt]
\bf Link Tag & \bf Predicate Argument Structure \\\midrule
PART\_OF   &    \textsc{part\_of}(from=text\#1, to=text\#2)    \\
OPINION & \textsc{opinion}(from=text\#1, to=text\#2, relation=opinion\_type)  \\
COREFERENCE &   \textsc{coreference}(from=text\#1, to=text\#2) \\
\bottomrule[1.25pt]
\end {tabular}\par
\end{minipage}

\paragraph{}
The following is an example of how the format is utilized:
For example:
\begin{enumerate}
\item I had a \textbf{\textcolor{blue}{Shrimp dish}} that was \textbf{\textcolor{red}{out of this world$_{positive}$}} and \textbf{\textcolor{yellow}{it}} was served over \textbf{\textcolor{red}{amazing$_{positive}$ fresh$_{positive}$}} \textbf{\textcolor{blue}{veggies}}. \\
\textsc{part\_of}(from=\textbf{\textcolor{blue}{veggies}}, to=\textbf{\textcolor{yellow}{it}}) \\
\textsc{coreference}(from=\textbf{\textcolor{yellow}{it}}, to=\textbf{\textcolor{blue}{Shrimp dish}})\\
\textsc{opinion}(from=\textbf{\textcolor{red}{out of this world}}, to=\textbf{\textcolor{blue}{Shrimp dish}} , relation="positive")\\
\textsc{opinion}(from=\textbf{\textcolor{red}{amazing}}, to=\textbf{\textcolor{blue}{veggies}} , relation="positive")\\
\textsc{opinion}(from=\textbf{\textcolor{red}{fresh}}, to=\textbf{\textcolor{blue}{veggies}} , relation="positive")
\end{enumerate}

\newpage
\section{Extent Tags}

\subsection{FOOD}
\paragraph{}
The FOOD tag concerns with specific food or beverage names, i.e. corns, beans, burritos, street tacos, margaritas, etc. For example:
\begin{enumerate}[resume]
\item \textbf{\textcolor{blue}{Chips}} and \textbf{\textcolor{blue}{salsa}} were \textbf{\textcolor{red}{great$_{positive}$}}.\\
\textsc{opinion}(from=\textbf{\textcolor{red}{great}}, to=\textbf{\textcolor{blue}{Chips}} , relation="positive")\\
\textsc{opinion}(from=\textbf{\textcolor{red}{great}}, to=\textbf{\textcolor{blue}{salsa}} , relation="positive")
\end{enumerate}

\paragraph{}
One exception where a specific food name is not annotated is when there is no description for that dish or beverage, in terms of ingredients or quality. The following example is a complete review of a customer. In this review, a specific dish, \textit{Beach Burger}, is mentioned. It is not annotated with a FOOD tag, because there is no description for this dish.

\begin{enumerate}[resume]
\item Phew, this is a GREAT place for sure. The reviews are what got me here and no doubt I WILL be back. Had the Beach Burger.
\end{enumerate}

\paragraph{}
When the generic term ``food'' is used, it should not be annotated. In the following examples, the terms ``food'' and ``meal'' are not annotated with a FOOD tag, because they are not specific food names. 
\begin{enumerate}[resume]
\item The food here is great!
\item We stopped for our first meal in Phoenix and loved it.
\end{enumerate}


\subsection{QUALITY}



\subsection{ANAPHORA}



\newpage
\section{Link Tags}


\subsection{PART\_OF}


\subsection{OPINION}


\subsection{COREFERENCE}


\end{document}