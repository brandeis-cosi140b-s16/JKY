\documentclass{article}
\usepackage{enumitem}
\usepackage{color}
\usepackage{forest}
\usepackage{tikz-qtree}
\usepackage{booktabs}
\usepackage{changepage}
\usepackage{lipsum}
\usepackage[fleqn]{amsmath}
\usepackage{fancyhdr}
\usepackage[a4paper, portrait, margin=1in]{geometry}
\linespread{1.5}
\title{Yelp Reviews Annotation Guidelines -- Draft}
\author{Qishen ``Justin'' Su, Kelley Lynch, Yuanyuan Ma}
\begin{document}

\newpage
\pagenumbering{arabic}
\maketitle

\tableofcontents

\newpage
\section{Introduction}
\paragraph{}
This document provides annotation guidelines for the Yelp restaurant reviews annotation task. The Yelp restaurant reviews annotation task involves (1) annotating the relation between a specific dish mentioned in a review and its described quality or characteristics, (2) annotating the relation between a dish and its ingredients or its parts, and (3) annotating the anaphors of dishes. These guidelines provide details on how to annotate Yelp restaurant reviews. 
\paragraph{}
This document is organized in the following manner: Section 2 is focused on dialog, while Section 3 concerns with non-dialog discourse. Each section presents various types of connections between text messages with specific examples. Many of the types in Section 2 are adapted from the backward looking functions in the DAMSL framework (Dialog Act Markup in Several Layers), \footnote{Allen, J. \& Core, M. (1997, March 22). \textit{Draft of DAMSL: Dialog Act Markup in Several Layers}. Retrieved February 1, 2016, from https://www.cs.rochester.edu/research/speech/damsl/RevisedManual/node12.html}.
\paragraph{}
When examples of annotation are provided in this document, the \textit{id} of each example text message is marked in \textbf{\textcolor{blue}{blue}}; for example, id=``\textbf{\textcolor{blue}{m0001}}''. The \textit{time} field represents the time and date when a text message is sent. The \textit{participant id} is the identifier of a sender; different participants have different ids.
\paragraph{}
Then, the connection between two text messages are represented using a predicate argument structure, in the form of \textsc{connection}(from\_id=id\#1, to\_id=id\#2, relation=relation\_type), which indicates that the text message of which the id is id\#1 is connected to the text message with id\#2 in certain type of discourse relation.

\newpage
\section{Dialog}

\subsection{Agreement}
\paragraph{}
Agreement can be a positive or negative response of a participant to a certain proposal, request, statement or belief. In general, a participant could accept or reject certain proposal. Occasionally, a participant may have an uncertain response to a proposal or statement. Sections 3.1 through 3.3 defines each subcategory and provides corresponding examples. 

\subsubsection{Acceptance}
\paragraph{}
Acceptance can be positive responses to proposals, requests, or assertions. Common key words of acceptance are ``yes'', ``ok'', ``alright'', etc. For example, \textsc{connection}(from\_id=``\textbf{\textcolor{blue}{m0121}}'', to\_id=``\textbf{\textcolor{blue}{m0122}}'') shows that participant 152255 agrees with the assertion in \textbf{\textcolor{blue}{m0121}}.

\begin{enumerate}[resume]
\item id=``\textbf{\textcolor{blue}{m0119}}" time=``2015-02-03 21:37:05 UTC" participant=``152255"\\
I want to take you out and treat you\\
id=``\textbf{\textcolor{blue}{m0120}}" time=``2015-02-03 21:37:13 UTC" participant=``152212"\\
Hell no\\
id=``\textbf{\textcolor{blue}{m0121}}" time=``2015-02-03 21:37:17 UTC" participant=``152212"\\
but yes we need to hang out\\
id=``\textbf{\textcolor{blue}{m0122}}" time=``2015-02-03 21:37:20 UTC" participant=``152255"\\
Yes !\\
\textsc{connection}(from\_id=``\textbf{\textcolor{blue}{m0119}}'', to\_id=``\textbf{\textcolor{blue}{m0120}}'', relation = Rejection)\\
\textsc{connection}(from\_id=``\textbf{\textcolor{blue}{m0120}}'', to\_id=``\textbf{\textcolor{blue}{m0121}}'', relation = Elaboration)\\
\textsc{connection}(from\_id=``\textbf{\textcolor{blue}{m0121}}'', to\_id=``\textbf{\textcolor{blue}{m0122}}'', relation = Acceptance)
\end{enumerate}

\subsubsection{Rejection}
\paragraph{}
Rejection are negative responses to proposals, requests, or assertions. Rejects often includes words like ``no'' or ``nah''. In example 1 of Section 2.1.1, \textbf{\textcolor{blue}{m0120}} is a rejection to the proposal in \textbf{\textcolor{blue}{m0119}}. In example 2, message \textbf{\textcolor{blue}{m0030}} declines the proposal in \textbf{\textcolor{blue}{m0028}}, which is also a case of rejection.
\begin{enumerate}[resume]
\item id=``\textbf{\textcolor{blue}{m0028}}" time=``2015-02-03 21:02:56 UTC" participant=``152255"\\
Omg when does your class end? Maybe we can get a drink?\\
id=``\textbf{\textcolor{blue}{m0029}}" time=``2015-02-03 21:02:56 UTC" participant=``152255"\\
Or is that too late for you?\\
id=``\textbf{\textcolor{blue}{m0030}}" time=``2015-02-03 21:15:23 UTC" participant=``152212"\\
I won't be able to. I am sorry. I am pretty busy through february. I'm going away a lot. But if you'll be at school tomorrow, ill try to get there early enough to chat!\\
\textsc{connection}(from\_id=``\textbf{\textcolor{blue}{m0028}}'', to\_id=``\textbf{\textcolor{blue}{m0029}}'', relation = Alternative)\\
\textsc{connection}(from\_id=``\textbf{\textcolor{blue}{m0028}}'', to\_id=``\textbf{\textcolor{blue}{m0030}}'', relation = Rejection)\\
\end{enumerate}

\subsubsection{Maybe}
\paragraph{}
Maybe applies to non-committal responses to a proposal or assertion. Some example responses are ``maybe'', ``let me think about it'', etc, which may leave a proposal open. In the following example, participant 153417 does not directly accept or reject the request of participant 153419, and leaves the request open; therefore, this type of connection is considered as ``maybe''.

\begin{enumerate}[resume]
\item id=``\textbf{\textcolor{blue}{m0123}}" time=``2015-01-19 16:28:15 UTC" participant=``153419"\\
Can you proofread my thing for me this weekend? Lol i think it sucks. It needs some crystal pizzazz \\
id=``\textbf{\textcolor{blue}{m0124}}" time=``2015-01-19 17:07:12 UTC" participant=``153417"\\
You silly\\
\textsc{connection}(from\_id=``\textbf{\textcolor{blue}{m0123}}'', to\_id=``\textbf{\textcolor{blue}{m0124}}'', relation = Maybe)\\
\end{enumerate}

\subsection{Understanding}
\paragraph{}
This type of responses concerns if a participant understands a statement from a previous message, and the examples of signaling understanding and non-understanding are given in the following subsections.

\subsubsection{Acknowledgment}
\paragraph{}
Acknowledgments signal a participant's understanding of a previous message. Often, acknowledgments consist of words or phrases like ``ok'', ``I understand'', ``yes'', etc.(see example 4). Sometimes, an acknowledgement can interrupt continuous text messages (see example 5). 

\begin{enumerate}[resume]
\item id=``\textbf{\textcolor{blue}{m0123}}" time=112015-02-03 21:37:32 UTC" participant=11152212"\\
I’m just in a bit of an overwhelmed state (which is my status quo)\\
id=``\textbf{\textcolor{blue}{m0124}}" time=``2015-02-03 21:37:38 UTC" participant=``152255"\\
Yeah I know\\
\textsc{connection}(from\_id=``\textbf{\textcolor{blue}{m0123}}'', to\_id=``\textbf{\textcolor{blue}{m0124}}'', relation = Acknowledgment)\\
\item id=``\textbf{\textcolor{blue}{m0061}}" time=``2015-02-03 21:28:32 UTC" participant=``152212"\\
you need to trust him\\
id=``\textbf{\textcolor{blue}{m0062}}" time="2015-02-03 21:28:32 UTC" participant=``152255"\\
Ok\\
id=``\textbf{\textcolor{blue}{m0063}}" time=``2015-02-03 21:28:32 UTC" participant=``152212"\\
or you’ll drive yourself crazy\\
\textsc{connection}(from\_id=``\textbf{\textcolor{blue}{m0061}}'', to\_id=``\textbf{\textcolor{blue}{m0062}}'', relation = Acknowledgment)\\
\textsc{connection}(from\_id=``\textbf{\textcolor{blue}{m0061}}'', to\_id=``\textbf{\textcolor{blue}{m0063}}'', relation = Alternative)
\end{enumerate}

\subsubsection{Non-Understanding}
\paragraph{}
Non-understanding is often signals by clarification questions. Note that not all clarification questions signals non-understanding, and to test if a text message signals non-understanding, it should be able to be phrased as ``what did you say'' or ``what did you mean'' (Allen 1997). The following are the examples of non-understanding. 

\begin{enumerate}[resume]
\item id=``\textbf{\textcolor{blue}{m0209}}" time=``2015-02-04 01:45:39 UTC" participant=``152212"\\
yeah i think he might be danger of losing his job\\
id=``\textbf{\textcolor{blue}{m0210}}" time=``2015-02-04 01:45:48 UTC" participant=``152252"\\
Really?? \\
\textsc{connection}(from\_id=``\textbf{\textcolor{blue}{m0209}}'', to\_id=``\textbf{\textcolor{blue}{m0210}}'', relation = Non-understanding)
\item id=``\textbf{\textcolor{blue}{m0086}}" time=``2015-02-09 20:30:37 UTC" participant=``152115"\\
And we have a city they were born in.\\
id=``\textbf{\textcolor{blue}{m0087}}" time=``2015-02-09 20:30:45 UTC" participant=``152115"\\
John and Katherine\\
id=``\textbf{\textcolor{blue}{m0088}}" time=``2015-02-09 20:31:35 UTC" participant=``152116"\\
What\\
id=``\textbf{\textcolor{blue}{m0089}}" time=``2015-02-09 20:32:04 UTC" participant=``152116"\\
They were born in the same place ?\\
\textsc{connection}(from\_id=``\textbf{\textcolor{blue}{m0086}}'', to\_id=``\textbf{\textcolor{blue}{m0087}}'', relation = Clarification)\\
\textsc{connection}(from\_id=``\textbf{\textcolor{blue}{m0086}}'', to\_id=``\textbf{\textcolor{blue}{m0088}}'', relation = Non-understanding)\\
\textsc{connection}(from\_id=``\textbf{\textcolor{blue}{m0086}}'', to\_id=``\textbf{\textcolor{blue}{m0089}}'', relation = Question)
\end{enumerate}

\subsection{Directives}
This type of discourse relation is used when a participant believes the other participant can perform certain action and wants them to carry out such an action.

\subsubsection{Request}
The ``Request'' relation is used when a participant asks another participant to perform certain action. 
\begin{enumerate}[resume]
\item id=``\textbf{\textcolor{blue}{m0135}}" time=``2015-02-21 23:23:44 UTC" participant=``149389"\\
I obv won't hear from them until this week. You are in a hotel!? \\
id=``\textbf{\textcolor{blue}{m0136}}" time=``2015-02-21 23:25:52 UTC" participant=``149389"\\
Oh! Actually I just found the business card template too!\\
id=``\textbf{\textcolor{blue}{m0137}}" time=``2015-02-21 23:25:52 UTC" participant=``152212"\\
no I’m bot\\
id=``\textbf{\textcolor{blue}{m0138}}" time=``2015-02-21 23:25:52 UTC" participant=``152212"\\
call me\\
\textsc{connection}(from\_id=``\textbf{\textcolor{blue}{m0135}}'', to\_id=``\textbf{\textcolor{blue}{m0136}}'', relation = Change of Subject)\\
\textsc{connection}(from\_id=``\textbf{\textcolor{blue}{m0135}}'', to\_id=``\textbf{\textcolor{blue}{m0137}}'', relation = Answer)\\
\textsc{connection}(from\_id=``\textbf{\textcolor{blue}{m0135}}'', to\_id=``\textbf{\textcolor{blue}{m0138}}'', relation = Request)\\
\end{enumerate}

\subsubsection{Suggestion}
When a participant provides another participant an idea or plan for consideration of a future action, this type of discourse relation is considered as ``Suggestion''.
\begin{enumerate}[resume]
\item id=``\textbf{\textcolor{blue}{m0099}}" time=``2015-02-03 21:32:48 UTC" participant=``152255"\\
I'm working hard to trust him thoigh. Giving him space. \\
id=``\textbf{\textcolor{blue}{m0100}}" time=``2015-02-03 21:32:48 UTC" participant=``152212" \\
space and trust aren’t the same thin \\
id=``\textbf{\textcolor{blue}{m0101}}" time=``2015-02-03 21:32:48 UTC" participant=``152212" \\
thing \\
id=``\textbf{\textcolor{blue}{m0102}}" time=``2015-02-03 21:34:56 UTC" participant=``152212" \\
id ask yourself what you need\\
id=``\textbf{\textcolor{blue}{m0103}}" time="2015-02-03 21:34:56 UTC" participant="152212"\\
out of the relationship\\
\textsc{connection}(from\_id=``\textbf{\textcolor{blue}{m0099}}'', to\_id=``\textbf{\textcolor{blue}{m0100}}'', relation = Rejection)\\
\textsc{connection}(from\_id=``\textbf{\textcolor{blue}{m0100}}'', to\_id=``\textbf{\textcolor{blue}{m0101}}'', relation = Correction)\\
\textsc{connection}(from\_id=``\textbf{\textcolor{blue}{m0100}}'', to\_id=``\textbf{\textcolor{blue}{m0102}}'', relation = Suggestion)\\
\textsc{connection}(from\_id=``\textbf{\textcolor{blue}{m0102}}'', to\_id=``\textbf{\textcolor{blue}{m0103}}'', relation = Elaboration)
\end{enumerate}

\subsection{Question}
\paragraph{}
This type of connections concerns with requests of information and clarification. Unlike the clarification questions mentioned previously, this type of questions do not signal non-understanding, but generally request for additional information. Note that in example 11, although the response``Really???'' in \textbf{\textcolor{blue}{m0072}} is a clarification question, it does not signal that participant 152116 does not understand message \textbf{\textcolor{blue}{m0071}}. 

\begin{enumerate}[resume]
\item id=``\textbf{\textcolor{blue}{m0004}}" time=``2015-02-03 20:58:40 UTC" participant=``152212"\\
i started seeing a new therapist\\
id=``\textbf{\textcolor{blue}{m0006}}" time=``2015-02-03 20:58:40 UTC" participant=``152255"\\
How is it going?\\
id=``\textbf{\textcolor{blue}{m0007}}" time=``2015-02-03 20:58:40 UTC" participant=``152255"\\
Is it a better therapist? Are you on campus? \\
\textsc{connection}(from\_id=``\textbf{\textcolor{blue}{m0004}}'', to\_id=``\textbf{\textcolor{blue}{m0006}}'', relation = Question)\\
\textsc{connection}(from\_id=``\textbf{\textcolor{blue}{m0006}}'', to\_id=``\textbf{\textcolor{blue}{m0007}}'', relation = Question)\\
\item id=``\textbf{\textcolor{blue}{m0071}}" time=``2015-02-09 20:15:37 UTC" participant=``152115"\\
My mom wanted to call Joe earlier so she's talking to him now.\\
id=``\textbf{\textcolor{blue}{m0072}}" time=``2015-02-09 20:16:10 UTC" participant=``152116"\\
Really???\\
id=``\textbf{\textcolor{blue}{m0073}}" time=``2015-02-09 20:16:28 UTC" participant=``152116"\\
Does she have the questions?\\
id=``\textbf{\textcolor{blue}{m0074}}" time=``2015-02-09 20:16:36 UTC" participant=``152116"\\
Was she anxious?\\
\textsc{connection}(from\_id=``\textbf{\textcolor{blue}{m0071}}'', to\_id=``\textbf{\textcolor{blue}{m0072}}'', relation = Question)\\
\textsc{connection}(from\_id=``\textbf{\textcolor{blue}{m0071}}'', to\_id=``\textbf{\textcolor{blue}{m0073}}'', relation = Question)\\
\textsc{connection}(from\_id=``\textbf{\textcolor{blue}{m0071}}'', to\_id=``\textbf{\textcolor{blue}{m0074}}'', relation = Question)\\
\end{enumerate}

\subsection{Answer}
\paragraph{}
This kind of connections responses to the request of information of a previous message. An answer can be either complete or partial; or it can a ``hold'' with which a participant signals their acknowledgment of a question, but does not provide an answer to it.

\subsubsection{Answer}
\paragraph{}
An answer provides information to a question in a previous text message. In the following example, \textbf{\textcolor{blue}{m0014}} is the answer to the question in \textbf{\textcolor{blue}{m0011}}.

\begin{enumerate}[resume]
\item
id="m0011" time="2015-02-03 20:58:40 UTC" participant="152255"\\
What are you teaching? \\
id="m0012" time="2015-02-03 21:00:33 UTC" participant="152212"\\
good question\\
id="m0013" time="2015-02-03 21:00:35 UTC" participant="152212"\\
haha\\
id="m0014" time="2015-02-03 21:00:48 UTC" participant="152212"\\
no I’m getting paid to go in to a DMI class and discuss some readings\\
\textsc{connection}(from\_id=``\textbf{\textcolor{blue}{m0011}}'', to\_id=``\textbf{\textcolor{blue}{m0012}}'', relation = Hold)\\
\textsc{connection}(from\_id=``\textbf{\textcolor{blue}{m0012}}'', to\_id=``\textbf{\textcolor{blue}{m0013}}'', relation = Emotions)\\
\textsc{connection}(from\_id=``\textbf{\textcolor{blue}{m0011}}'', to\_id=``\textbf{\textcolor{blue}{m0014}}'', relation = Answer)\\
\end{enumerate}

\subsubsection{Hold}
\paragraph{}
A participant sometimes signals their acknowledgment of a question, but does not provide an answer to it. In example 12 of Section 2.5.1, message \textbf{\textcolor{blue}{m0012}} is considered as a ``hold'', since it addresses the question in \textbf{\textcolor{blue}{m0011}}, but does not provide an answer to it. The following example is another case of hold, where the question is not answered.

\begin{enumerate}[resume]
\item id=``\textbf{\textcolor{blue}{m0039}}" time=``2015-02-01 19:03:44 UTC" participant=``153419"\\
Yayy. Should i bring oliver next week?! Lol\\
id=``\textbf{\textcolor{blue}{m0040}}" time=``2015-02-01 19:04:26 UTC" participant=``153417"\\
Lol you crazy!\\
\textsc{connection}(from\_id=``\textbf{\textcolor{blue}{m0039}}'', to\_id=``\textbf{\textcolor{blue}{m0040}}'', relation = Hold)\\
\end{enumerate}

\subsection{Social Obligations}
This type of discourse relation is used when a participant complies with certain social norms or obligations, such as apologies or appreciation. 

\subsubsection{Apology}
\paragraph{}
This discourse relation is used when one participant apologizes when they realize a social norm is violated, and an apology could be either direct or indirect. 
\begin{enumerate}[resume]
\item id=``\textbf{\textcolor{blue}{m0078}}" time=``2015-02-09 20:18:56 UTC" participant=``152116"\\
I cant take This you know that\\
id=``\textbf{\textcolor{blue}{m0078}}" time=``2015-02-09 20:19:25 UTC" participant=``152115"\\
I don't know! I can't hear! Haha. Sorry\\
\textsc{connection}(from\_id=``\textbf{\textcolor{blue}{m0078}}'', to\_id=``\textbf{\textcolor{blue}{m0079}}'', relation = Apology)\\
\end{enumerate}

\subsubsection{Gratitude}
\paragraph{}
This relation is used when a participant expresses their gratitude or appreciation to another participant.
\begin{enumerate}[resume]
\item id=``\textbf{\textcolor{blue}{m0109}}'' time=``2015-02-03 21:34:56 UTC" participant=``152212"\\
your a very special woman, a very beautiful woman. A supportive and understanding partner. he’s lucky to have found you.\\
id=``\textbf{\textcolor{blue}{m0110}}'' time=``2015-02-03 21:34:56 UTC" participant=``152255"\\
Awwww omg \\
id=``\textbf{\textcolor{blue}{m0111}}'' time=``2015-02-03 21:34:56 UTC" participant=``152212"\\
None of the insecurities you are expressing are weird. \\
id=``\textbf{\textcolor{blue}{m0112}}'' time=``2015-02-03 21:34:56 UTC" participant=``152255"\\
Thank you! I needed to hear that\\
\textsc{connection}(from\_id=``\textbf{\textcolor{blue}{m0109}}'', to\_id=``\textbf{\textcolor{blue}{m0110}}'', relation = Emotions)\\
\textsc{connection}(from\_id=``\textbf{\textcolor{blue}{m0109}}'', to\_id=``\textbf{\textcolor{blue}{m0111}}'', relation = Elaboration)\\
\textsc{connection}(from\_id=``\textbf{\textcolor{blue}{m0109}}'', to\_id=``\textbf{\textcolor{blue}{m0112}}'', relation = Thanks)\\
\end{enumerate}


\subsection{Other}
\paragraph{}
Some types of responses are not formally categorized into any one of the previous types; however, it is important to show some representative examples of certain types of responses for annotation purposes

\subsubsection{Emotions}
\paragraph{}
A participant may respond to a text message with emotional words or phrases, like laughing words (such as ``haha'' and ``lol''), surprise or excitement words (such as ``omg'' or ``yay''), appreciation words or phrases (such as ``awww'' and ``thank you''), or emoticons. In example 16, \textbf{\textcolor{blue}{m0110}} shows the appreciation of participant 152255 toward the compliments in  \textbf{\textcolor{blue}{m0109}}. Example 17 shows two instances of excitement: connection between \textbf{\textcolor{blue}{m0126}} and \textbf{\textcolor{blue}{m0127}}, and connection between \textbf{\textcolor{blue}{m0128}} and \textbf{\textcolor{blue}{m0129}}. Note that \textbf{\textcolor{blue}{m0126}} and \textbf{\textcolor{blue}{m0128}} are connected because they are two continuous text messages, and \textbf{\textcolor{blue}{m0128}} does not respond to \textbf{\textcolor{blue}{m0127}}. 
\begin{enumerate}[resume]
\item id=``\textbf{\textcolor{blue}{m0109}}" time=``2015-02-03 21:34:56 UTC" participant=``152212" \\
your a very special woman, a very beautiful woman. A supportive and understanding partner. he's lucky to have found you.\\
id=``\textbf{\textcolor{blue}{m0110}}" time=``2015-02-03 21:34:56 UTC"  participant=``152255" \\
Awwww omg \\
\textsc{connection}(from\_id=``\textbf{\textcolor{blue}{m0109}}'', to\_id=``\textbf{\textcolor{blue}{m0110}}'', relation = Emotions)\\
\item id=``\textbf{\textcolor{blue}{m0126}}" time=``2015-02-03 21:37:52 UTC" participant=``152212" \\
nate is also coming \\
id=``\textbf{\textcolor{blue}{m0127}}" time=``2015-02-03 21:38:05 UTC" participant=``152255" \\
Omg!!! Yaya\\
id=``\textbf{\textcolor{blue}{m0128}}" time=``2015-02-03 21:38:05 UTC" participant=``152212"\\
but i decided i didn't want to do a party again\\
id=``\textbf{\textcolor{blue}{m0129}}" time=``2015-02-03 21:38:06 UTC" participant=``152255"\\
Yay\\
\textsc{connection}(from\_id=``\textbf{\textcolor{blue}{m0126}}'', to\_id=``\textbf{\textcolor{blue}{m0127}}'', relation = Emotions)\\
\textsc{connection}(from\_id=``\textbf{\textcolor{blue}{m0128}}'', to\_id=``\textbf{\textcolor{blue}{m0129}}'', relation = Emotions)\\
\textsc{connection}(from\_id=``\textbf{\textcolor{blue}{m0126}}'', to\_id=``\textbf{\textcolor{blue}{m0128}}'', relation = Elaboration)\\
\item id=``\textbf{\textcolor{blue}{m0079}}" time=``2015-02-09 20:19:25 UTC" participant=``152115"\\
I don't know! I can't hear! Haha. Sorry\\
id=``\textbf{\textcolor{blue}{m0080}}" time=``2015-02-09 20:20:10 UTC" participant=``152116"\\
D:\\
\textsc{connection}(from\_id=``\textbf{\textcolor{blue}{m0079}}'', to\_id=``\textbf{\textcolor{blue}{m0080}}'', relation = Emotions)\\
\end{enumerate}

\newpage
\section{Non-Dialog}

\subsection{Contingency}
\paragraph{}
This type of discourse relations is used to indicates the situations described in two or more text messages causally influence each other.

\subsubsection{Cause}
\paragraph{}
Causes indicate that the information or situations in two text messages influence each other causally, and they are not in a conditional relation (Penn Discourse Treebank 2007). This type of relation is used when the argument of a previous message is the cause, and that of a latter message is the result or effect. 
\begin{enumerate}[resume]
\item id=``\textbf{\textcolor{blue}{m0040}}" time=``2015-02-04 00:46:53 UTC" participant=``152252"\\
There is no street parking\\
id=``\textbf{\textcolor{blue}{m0041}}" time=``2015-02-04 00:46:56 UTC" participant=``152252"\\
So I gave to leave work early \\
\textsc{connection}(from\_id=``\textbf{\textcolor{blue}{m0040}}'', to\_id=``\textbf{\textcolor{blue}{m0041}}'', relation = Cause)\\
\end{enumerate}

\subsubsection{Result}
\paragraph{}
Similar to the ``Cause'' relation, results also indicates that two arguments have a causal relation, and that they are not in a conditional relation. ``Result'' is used when the argument of a previous message is the result caused by the situation of a latter message.
\begin{enumerate}[resume]
\item id=``\textbf{\textcolor{blue}{m0282}}" time=``2015-02-04 02:44:16 UTC" participant=``152212"\\
it’s just unclear\\
id=``\textbf{\textcolor{blue}{m0283}}" time=``2015-02-04 02:46:24 UTC" participant=``152212"\\
bc they only had one class so im not sure what kind of flow they have\\
\textsc{connection}(from\_id=``\textbf{\textcolor{blue}{m0282}}'', to\_id=``\textbf{\textcolor{blue}{m0283}}'', relation = Result)\\
\end{enumerate}

\subsubsection{Condition}
\paragraph{}
Two text messages are in a conditional relation when the argument of one message is the condition and that of the other message is the consequence. The order of conditional relation does not matter; in other words, the condition can either precede or succeed the consequence. The following examples show two different orderings of condition and consequence.
\begin{enumerate}[resume]
\item id=``\textbf{\textcolor{blue}{m0125}}" time=``2015-02-04 01:12:32 UTC" participant=``152212"\\
i actually would suggest you rent a garage for the month\\
id=``\textbf{\textcolor{blue}{m0126}}" time=``2015-02-04 01:12:32 UTC" participant=``152212"\\
especially if you are saving money on gas\\
\textsc{connection}(from\_id=``\textbf{\textcolor{blue}{m0125}}'', to\_id=``\textbf{\textcolor{blue}{m0126}}'', relation = Condition)\\
\item id=``m0263" time=``2015-02-20 13:25:06 UTC" participant=``152212"\\
also if you and alec can pose next to a baby carriage\\
id=``m0264" time=``2015-02-20 13:25:10 UTC" participant=``152212"\\
or a moped\\
id=``m0265" time=``2015-02-20 13:25:12 UTC" participant=``152212"\\
anything funny\\
id=``m0266" time=``2015-02-20 13:25:21 UTC" participant=``152212"\\
i promise ill make you both iron on t-shirts\\
\textsc{connection}(from\_id=``\textbf{\textcolor{blue}{m0263}}'', to\_id=``\textbf{\textcolor{blue}{m0264}}'', relation = Elaboration)\\
\textsc{connection}(from\_id=``\textbf{\textcolor{blue}{m0263}}'', to\_id=``\textbf{\textcolor{blue}{m0265}}'', relation = Alternative)\\
\textsc{connection}(from\_id=``\textbf{\textcolor{blue}{m0263}}'', to\_id=``\textbf{\textcolor{blue}{m0266}}'', relation = Condition)\\
\end{enumerate}

\subsection{Expansion}
\paragraph{}
This type of relations concerns with the expansion and continuity of discourse. It can be completing sentences, proceeding a conversation, clarification, or supplying additional information.

\subsubsection{Elaboration}
\paragraph{} 
A text message is considered as elaboration of a previous one, when the current message completes or elaborates on the information that the previous message conveys. In the following example, \textbf{\textcolor{blue}{m0004}} is considered as an elaboration of \textbf{\textcolor{blue}{m0003}}. 

\begin{enumerate}[resume]
\item id=``\textbf{\textcolor{blue}{m0001}}" time="2015-02-03 20:54:24 UTC" participant="152212"\\
i can't come tonight. I'm prepping for a class on wednesday night that I'm teaching\\
id=``\textbf{\textcolor{blue}{m0002}}" time="2015-02-03 20:58:27 UTC" participant="152212"\\
but thank you for the invite\\
id=``\textbf{\textcolor{blue}{m0003}}" time="2015-02-03 20:58:37 UTC" participant="152212"\\
I'm a bit crazy if you haven't noticed\\
id=``\textbf{\textcolor{blue}{m0004}}" time="2015-02-03 20:58:40 UTC" participant="152212"\\
i started seeing a new therapist\\
\textsc{connection}(from\_id=``\textbf{\textcolor{blue}{m0001}}'', to\_id=``\textbf{\textcolor{blue}{m0002}}'', relation = Elaboration)\\
\textsc{connection}(from\_id=``\textbf{\textcolor{blue}{m0002}}'', to\_id=``\textbf{\textcolor{blue}{m0003}}'', relation = Change of Subject)\\
\textsc{connection}(from\_id=``\textbf{\textcolor{blue}{m0003}}'', to\_id=``\textbf{\textcolor{blue}{m0004}}'', relation = Elaboration)
\item id=``\textbf{\textcolor{blue}{m0037}}" time=``2015-02-03 21:24:16 UTC" participant=``152212" \\
i think it is just post graduation. Like i have all these little gigs \\
id=``\textbf{\textcolor{blue}{m0038}}" time=``2015-02-03 21:24:16 UTC" participant=``152212" \\
and i nothing is guaranteed \\
id=``\textbf{\textcolor{blue}{m0039}}" time=``2015-02-03 21:24:16 UTC" participant=``152212" \\
and I'm constantly needing to prove myself \\
id=``\textbf{\textcolor{blue}{m0040}}" time=``2015-02-03 21:24:16 UTC" participant=``152255" \\
Yeah I've always heard the first two years are the toughest \\
\textsc{connection}(from\_id=``\textbf{\textcolor{blue}{m0037}}'', to\_id=``\textbf{\textcolor{blue}{m0038}}'', relation = Elaboration)\\
\textsc{connection}(from\_id=``\textbf{\textcolor{blue}{m0038}}'', to\_id=``\textbf{\textcolor{blue}{m0039}}'', relation = Elaboration)\\
\textsc{connection}(from\_id=``\textbf{\textcolor{blue}{m0037}}'', to\_id=``\textbf{\textcolor{blue}{m0040}}'', relation = Agreement)
\end{enumerate}

\subsubsection{Clarification}
\paragraph{}
The ``Clarification'' relation is used when a text message clarifies or supply additional information for a previous message to avoid confusion. In example 25, \textbf{\textcolor{blue}{m0087}} clarifies to whom ``they'' in \textbf{\textcolor{blue}{m0086}} refers. 

\begin{enumerate}[resume]
\item id=``\textbf{\textcolor{blue}{m0086}}" time=``2015-02-09 20:30:37 UTC" participant=``152115"\\
And we have a city they were born in.\\
id=``\textbf{\textcolor{blue}{m0087}}" time=``2015-02-09 20:30:45 UTC" participant=``152115"\\
John and Katherine\\
\textsc{connection}(from\_id=``\textbf{\textcolor{blue}{m0086}}'', to\_id=``\textbf{\textcolor{blue}{m0087}}'', relation = Clarification)
\end{enumerate}

\subsection{Correction}
\paragraph{}
Corrections generally concern with correcting wrong information from a previous text message, such as typos. Any related following messages should connect to the message with typos. In the example below, \textbf{\textcolor{blue}{m0101}} corrects the typo in \textbf{\textcolor{blue}{m0100}}; thus, there is a connection between these two messages. Then, \textbf{\textcolor{blue}{m0100}} should connect to \textbf{\textcolor{blue}{m0103}}.

\begin{enumerate}[resume]
\item id=``\textbf{\textcolor{blue}{m0099}}" time=``2015-02-03 21:32:48 UTC" participant=``152255"\\
I'm working hard to trust him thoigh. Giving him space. \\
id=``\textbf{\textcolor{blue}{m0100}}" time=``2015-02-03 21:32:48 UTC" participant=``152212" \\
space and trust aren’t the same thin \\
id=``\textbf{\textcolor{blue}{m0101}}" time=``2015-02-03 21:32:48 UTC" participant=``152212" \\
thing \\
id=``\textbf{\textcolor{blue}{m0102}}" time=``2015-02-03 21:34:56 UTC" participant=``152212" \\
id ask yourself what you need\\
id=``\textbf{\textcolor{blue}{m0103}}" time="2015-02-03 21:34:56 UTC" participant="152212"\\
out of the relationship\\
\textsc{connection}(from\_id=``\textbf{\textcolor{blue}{m0099}}'', to\_id=``\textbf{\textcolor{blue}{m0100}}'', relation = Rejection)\\
\textsc{connection}(from\_id=``\textbf{\textcolor{blue}{m0100}}'', to\_id=``\textbf{\textcolor{blue}{m0101}}'', relation = Correction)\\
\textsc{connection}(from\_id=``\textbf{\textcolor{blue}{m0100}}'', to\_id=``\textbf{\textcolor{blue}{m0102}}'', relation = Suggestion)\\
\textsc{connection}(from\_id=``\textbf{\textcolor{blue}{m0102}}'', to\_id=``\textbf{\textcolor{blue}{m0103}}'', relation = Elaboration)
\end{enumerate}

\subsection{Concession}
\paragraph{}
This type of discourse relation is used to highlight prominent differences between the arguments in two text messages. More specifically, ``the highlighted differences are related to expectations raised by one argument which are then denied by the other'' (Penn Discourse TreeBank 2007).
\begin{enumerate}[resume]
\item id=``\textbf{\textcolor{blue}{m0157}}'' time=``2015-02-22 01:32:55 UTC" participant=``149389"\\
I should have sent you pics!\\
id=``\textbf{\textcolor{blue}{m0158}}'' time=``2015-02-22 01:33:04 UTC" participant=``149389"\\
Though I guess it is nothing compared to boston \\
\textsc{connection}(from\_id=``\textbf{\textcolor{blue}{m0157}}'', to\_id=``\textbf{\textcolor{blue}{m0158}}'', relation = Concession)\\
\end{enumerate}

\subsection{Alternative}
\paragraph{}
This discourse relation is used when two text messages describe alternative situations. Words like ``or'', ``instead'' and ``otherwise'' may be helpful to identify an ``Alternative'' relation. 
\begin{enumerate}[resume]
\item id=``\textbf{\textcolor{blue}{m0157}}'' time=``2015-02-04 01:33:21 UTC" participant=``152212"\\
i wouldn’t mind putting 200\$ on a cc for a month \\
id=``\textbf{\textcolor{blue}{m0158}}'' time=``2015-02-04 01:33:23 UTC" participant=``152212"\\
to put the car away\\
id=``\textbf{\textcolor{blue}{m0159}}'' time=``2015-02-04 01:33:25 UTC" participant=``152212"\\
or two weeks\\
id=``\textbf{\textcolor{blue}{m0160}}'' time=``2015-02-04 01:33:28 UTC" participant=``152212"\\
or whatever\\
id=``\textbf{\textcolor{blue}{m0161}}'' time=``2015-02-04 01:33:34 UTC" participant=``152212"\\
bc it bothers you so much\\
\textsc{connection}(from\_id=``\textbf{\textcolor{blue}{m0157}}'', to\_id=``\textbf{\textcolor{blue}{m0158}}'', relation = Elaboration)\\
\textsc{connection}(from\_id=``\textbf{\textcolor{blue}{m0157}}'', to\_id=``\textbf{\textcolor{blue}{m0159}}'', relation = Alternative)\\
\textsc{connection}(from\_id=``\textbf{\textcolor{blue}{m0157}}'', to\_id=``\textbf{\textcolor{blue}{m0160}}'', relation = Alternative)\\
\textsc{connection}(from\_id=``\textbf{\textcolor{blue}{m0157}}'', to\_id=``\textbf{\textcolor{blue}{m0161}}'', relation = Result)\\
\end{enumerate}

\subsection{Other}
\paragraph{}
Some types of responses are not formally categorized into any one of the previous types; however, it is important to show some representative examples of certain types of responses for annotation purposes.

\subsubsection{Change of Subject}
\paragraph{}
It is considered a change of subject when the current text message does not relate to a previous message directly in terms of the information that both messages convey, regardless of time lapse between two messages. In the first following example, \textsc{connection}(from\_id=``\textbf{\textcolor{blue}{m0002}}'', to\_id=``\textbf{\textcolor{blue}{m0003}}'') shows message \textbf{\textcolor{blue}{m0002}} is connected to message \textbf{\textcolor{blue}{m0003}}, because of the short time span between them. In the second example , message \textbf{\textcolor{blue}{m0145}} does not relate to message \textbf{\textcolor{blue}{m0146}}, because the time difference is large enough to consider that message \textbf{\textcolor{blue}{m0146}} initiates a new conversion. Therefore, there is no connection between message \textbf{\textcolor{blue}{m0145}} and message \textbf{\textcolor{blue}{m0146}}.

\begin{enumerate}[resume]
\item id=``\textbf{\textcolor{blue}{m0001}}" time="2015-02-03 20:54:24 UTC" participant="152212"\\
i can't come tonight. I'm prepping for a class on wednesday night that I'm teaching\\
id=``\textbf{\textcolor{blue}{m0002}}" time="2015-02-03 20:58:27 UTC" participant="152212"\\
but thank you for the invite\\
id=``\textbf{\textcolor{blue}{m0003}}" time="2015-02-03 20:58:37 UTC" participant="152212"\\
I'm a bit crazy if you haven't noticed\\
id=``\textbf{\textcolor{blue}{m0004}}" time="2015-02-03 20:58:40 UTC" participant="152212"\\
i started seeing a new therapist\\
\textsc{connection}(from\_id=``\textbf{\textcolor{blue}{m0001}}'', to\_id=``\textbf{\textcolor{blue}{m0002}}'', relation = Elaboration)\\
\textsc{connection}(from\_id=``\textbf{\textcolor{blue}{m0002}}'', to\_id=``\textbf{\textcolor{blue}{m0003}}'', relation = Change of Subject)\\
\textsc{connection}(from\_id=``\textbf{\textcolor{blue}{m0003}}'', to\_id=``\textbf{\textcolor{blue}{m0004}}'', relation = Elaboration)
\item id=``\textbf{\textcolor{blue}{m0145}}" time="2015-02-03 21:43:00 UTC" participant="152212"\\
its silly stuff\\
id=``\textbf{\textcolor{blue}{m0146}}" time="2015-02-04 17:35:50 UTC" participant="152255"\\
Oh man I got drunk last night and just bombarded Jeff with weird texts about the potluck I was having. I apologized this morning and no response.\\
\textsc{connection}(from\_id=``\textbf{\textcolor{blue}{m0145}}'', to\_id=``\textbf{\textcolor{blue}{m0146}}'', relation = Change of Subject)
\end{enumerate}


\subsubsection{Punctuation}
\paragraph{}
Sometimes, a final punctuation of a sentence might be missing, and then supplied in a following message. It is necessary to connect these two messages to retain the semantics of the first sentence. In the following example, the connection between \textbf{\textcolor{blue}{m0131}} and \textbf{\textcolor{blue}{m0132}} is important to show that \textbf{\textcolor{blue}{m0131}} is actually a question, rather than a statement.

\begin{enumerate}[resume]
\item id=``\textbf{\textcolor{blue}{m0131}}'" time=``2015-02-09 21:21:46 UTC" participant=``152116"\\
And nothing on Katherine\\
id=``\textbf{\textcolor{blue}{m0132}}'" time=``2015-02-09 21:21:51 UTC" participant=``152116"\\
?\\
id=``\textbf{\textcolor{blue}{m0133}}'" time=``2015-02-09 21:22:19 UTC" participant=``152115"\\
Nope\\
\textsc{connection}(from\_id=``\textbf{\textcolor{blue}{m0131}}'', to\_id=``\textbf{\textcolor{blue}{m0132}}'', relation = Punctuation)\\
\textsc{connection}(from\_id=``\textbf{\textcolor{blue}{m0131}}'', to\_id=``\textbf{\textcolor{blue}{m0133}}'', relation = Answer)
\end{enumerate}


\end{document}