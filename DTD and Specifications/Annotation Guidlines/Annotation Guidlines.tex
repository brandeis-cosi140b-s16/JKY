\documentclass{article}
\usepackage{enumitem}
\usepackage{color}
\usepackage{forest}
\usepackage{tikz-qtree}
\usepackage{booktabs}
\usepackage{changepage}
\usepackage{lipsum}
\usepackage[fleqn]{amsmath}
\usepackage{fancyhdr}
\usepackage{tabularx,ragged2e,booktabs,caption}
\newcolumntype{C}[1]{>{\Centering}m{#1}}
\renewcommand\tabularxcolumn[1]{C{#1}}
\usepackage[a4paper, portrait, margin=1in]{geometry}
\linespread{1.5}
\title{Yelp Reviews Annotation Guidelines -- Draft}
\author{Qishen ``Justin'' Su, Kelley Lynch, Yuanyuan Ma}
\begin{document}

\newpage
\pagenumbering{arabic}
\maketitle

\tableofcontents

\newpage
\section{Introduction}
\paragraph{}
This document provides annotation guidelines for the Yelp restaurant reviews annotation task. The Yelp restaurant reviews annotation task involves (1) annotating the relation between a specific dish mentioned in a review and its described quality or characteristics, (2) annotating the relation between a dish and its ingredients or its parts, and (3) annotating the anaphors of dishes. These guidelines provide details on how to annotate Yelp restaurant reviews. 
\paragraph{}
This document is organized in the following manner: Section 2 is focused on annotating extent tags, i.e. FOOD, QUALITY and ANAPHORA, while Section 3 is focusing on annotating link tags, i.e. PART\_OF, OPINION, and COFERENCE. 
\paragraph{}
When examples of annotation are provided in this document, texts with different extent tags are marked in different colors: \textbf{\textcolor{blue}{blue}} is for FOOD, \textbf{\textcolor{red}{red}} is for QUALITY, and \textbf{\textcolor{yellow}{yellow}} is for ANAPHORA. A QUALITY tag contains three types: \textit{positive}, \textit{negative}, and \textit{neutral}, and in this document, a type of QUALITY will be presented in subscript. Then, each link tag is represented using a predicate argument structure given in Table 1. 
\\
\begin{minipage}{\linewidth}
\centering
\captionof{table}{Link Tag Predicate Argument} \label{tab:title} 
\begin{tabular}{ll}\toprule[1.5pt]
\bf Link Tag & \bf Predicate Argument Structure \\\midrule
PART\_OF   &    \textsc{part\_of}(from=text\#1, to=text\#2)    \\
OPINION & \textsc{opinion}(from=text\#1, to=text\#2, relation=opinion\_type)  \\
COREFERENCE &   \textsc{coreference}(from=text\#1, to=text\#2) \\
\bottomrule[1.25pt]
\end {tabular}\par
\end{minipage}

\paragraph{}
The following is an example of how the format is utilized:
For example:
\begin{enumerate}
\item I had a \textbf{\textcolor{blue}{Shrimp dish}} that was \textbf{\textcolor{red}{out of this world$_{positive}$}} and \textbf{\textcolor{yellow}{it}} was served over \textbf{\textcolor{red}{amazing$_{positive}$ fresh$_{positive}$}} \textbf{\textcolor{blue}{veggies}}. \\
\textsc{part\_of}(from=\textbf{\textcolor{blue}{veggies}}, to=\textbf{\textcolor{yellow}{it}}) \\
\textsc{coreference}(from=\textbf{\textcolor{yellow}{it}}, to=\textbf{\textcolor{blue}{Shrimp dish}})\\
\textsc{opinion}(from=\textbf{\textcolor{red}{out of this world}}, to=\textbf{\textcolor{blue}{Shrimp dish}} , relation=``positive")\\
\textsc{opinion}(from=\textbf{\textcolor{red}{amazing}}, to=\textbf{\textcolor{blue}{veggies}} , relation=``positive")\\
\textsc{opinion}(from=\textbf{\textcolor{red}{fresh}}, to=\textbf{\textcolor{blue}{veggies}} , relation=``positive")
\end{enumerate}

\newpage
\section{Extent Tags}

\subsection{FOOD}
\paragraph{}
The FOOD tag concerns with specific food or beverage names, i.e. corns, beans, burritos, street tacos, margaritas, etc. When annotating a food name, articles (i.e. \textit{a} and \textit{the}) should be excluded. For example:
\begin{enumerate}[resume]
\item The \textbf{\textcolor{blue}{chips}} and \textbf{\textcolor{blue}{salsa}} were \textbf{\textcolor{red}{great$_{positive}$}}.\\
\textsc{opinion}(from=\textbf{\textcolor{red}{great}}, to=\textbf{\textcolor{blue}{chips}} , relation=``positive")\\
\textsc{opinion}(from=\textbf{\textcolor{red}{great}}, to=\textbf{\textcolor{blue}{salsa}} , relation=``positive")
\end{enumerate}

\paragraph{}
One exception where a specific food name is not annotated is when there is no description for that dish or beverage, in terms of ingredients or quality. The following example is a complete review of a customer. In this review, a specific dish, \textit{Beach Burger}, is mentioned. It is not annotated with a FOOD tag, because there is no description for this dish.

\begin{enumerate}[resume]
\item Phew, this is a GREAT place for sure. The reviews are what got me here and no doubt I WILL be back. Had the Beach Burger.
\end{enumerate}

\paragraph{}
When the generic term ``food'' is used, it should not be annotated. In the following examples, the terms ``food'' and ``meal'' are not annotated with a FOOD tag, because they are not specific food names. 
\begin{enumerate}[resume]
\item The food here is great!
\item We stopped for our first meal in Phoenix and loved it.
\end{enumerate}


\subsection{QUALITY}
\paragraph{}


\subsection{ANAPHORA}
\paragraph{}
The anaphora tag is to mark names or pronouns that refers to the food items or entities mentioned in previous sentences or paragraphs. An anaphora could be a pronoun like "it", "they", "both", or another name for the food. The use of anaphora enables the link between quality and food be in the same sentence or clause, which will be discussed in "OPINION" session. Some examples of anaphoras are listed below:

\begin{enumerate}[resume]
	\item I had the \textbf{\textcolor{blue}{carne asada potato with cheese}} and oh my \textbf{\textcolor{yellow}{it}} was \textbf{\textcolor{red}{delicious$_{positive}$}}.\\
	\textsc{coreference}(from=\textbf{\textcolor{yellow}{it}}, to=\textbf{\textcolor{blue}{carne asada potato with cheese}})\\
		\textsc{opinion}(from=\textbf{\textcolor{red}{delicious}}, to=\textbf{\textcolor{yellow}{it}}, relation=``positive'')\\
\end{enumerate}

\begin{enumerate}[resume]
	\item We ordered two \textbf{\textcolor{blue}{agua frescas}}. One was  \textbf{\textcolor{blue}{mango}} and the other was 
	\textbf{\textcolor{blue}{cantaloupe/watermelon}} on the waitress' recommendation. Wow! 
	\textbf{\textcolor{yellow}{both}} were so 
	\textbf{\textcolor{red}{fresh$_{positive}$}} and 
	\textbf{\textcolor{red}{delicious$_{positive}$}}.\\
\end{enumerate}

\newpage
\section{Link Tags}


\subsection{PART\_OF}
\paragraph{}
The PART\_OF link tag describes a part-whole relation between an ingredient and a dish, or between a dish and a combo. The predicative argument structure \textsc{part\_of}(from=text\#1, to=text\#2) indicates, \textit{text\#1 is part of text\#2}. The following is an example of PART\_OF relation between ingredients and a dish.

\begin{enumerate}[resume]
\item I just couldn't believe my eyes when I saw all the stuff they squeezed into \textbf{\textcolor{blue}{both sandwiches}}! \textbf{\textcolor{blue}{Ham}}, \textbf{\textcolor{blue}{pork sirloin}}, \textbf{\textcolor{blue}{chorizo}}, \textbf{\textcolor{blue}{sausage}}, \textbf{\textcolor{blue}{egg}}, \textbf{\textcolor{blue}{avocado}}, \textbf{\textcolor{blue}{jalapenos}}, \textbf{\textcolor{blue}{breaded beef}}.\\
\textsc{part\_of}(from=\textbf{\textcolor{blue}{Ham}}, to=\textbf{\textcolor{blue}{both sandwiches}}) \\
\textsc{part\_of}(from=\textbf{\textcolor{blue}{pork sirloin}}, to=\textbf{\textcolor{blue}{both sandwiches}}) \\
\textsc{part\_of}(from=\textbf{\textcolor{blue}{chorizo}}, to=\textbf{\textcolor{blue}{both sandwiches}}) \\
\textsc{part\_of}(from=\textbf{\textcolor{blue}{sausage}}, to=\textbf{\textcolor{blue}{both sandwiches}}) \\
\textsc{part\_of}(from=\textbf{\textcolor{blue}{egg}} to=\textbf{\textcolor{blue}{both sandwiches}}) \\
\textsc{part\_of}(from=\textbf{\textcolor{blue}{avocado}}, to=\textbf{\textcolor{blue}{both sandwiches}}) \\
\textsc{part\_of}(from=\textbf{\textcolor{blue}{jalapenos}}, to=\textbf{\textcolor{blue}{both sandwiches}}) \\
\textsc{part\_of}(from=\textbf{\textcolor{blue}{breaded beef}}, to=\textbf{\textcolor{blue}{both sandwiches}}) \\
\end{enumerate}

\subsection{OPINION}
\paragraph{}

\subsection{COREFERENCE}
\paragraph{}
The CORREFERENCE tag links the anaphora with previous mention of the same entity. It serves, in addition, as the "to" object of the OPNION tag. Examples are as follows:

\begin{enumerate}[resume]
	\item I had the \textbf{\textcolor{blue}{carne asada potato}} with \textbf{\textcolor{blue}{cheese}} and oh my \textbf{\textcolor{yellow}{it}} was \textbf{\textcolor{red}{delicious$_{positive}$}}.\\
	
	\textsc{ANAPHORA}(text=\textbf{\textcolor{yellow}{it}})\\
	
	\textsc{COREFERENCE}(from=\textbf{\textcolor{blue}{it}} to=\textbf{\textcolor{blue}{carne asada potato}}) \\
	

\end{enumerate}

\begin{enumerate}[resume]
	\item I had the \textbf{\textcolor{blue}{carne asada potato}} with \textbf{\textcolor{blue}{cheese}} and oh my \textbf{\textcolor{yellow}{it}} was \textbf{\textcolor{red}{delicious$_{positive}$}}.\\
	
	\textsc{ANAPHORA}(text=\textbf{\textcolor{yellow}{it}})\\
	
	\textsc{COREFERENCE}(from=\textbf{\textcolor{blue}{it}} to=\textbf{\textcolor{blue}{carne asada potato}}) \\
\end{enumerate}

\paragraph{}
In the following example, the ANAPHORA \textbf{\textcolor{yellow}{both}} should be linked with the closest previous mentions of the entity, which is \textbf{\textcolor{blue}{mango}} and \textbf{\textcolor{blue}{cantaloupe/watermelon}} instead of \textbf{\textcolor{blue}{agua frescas}}.

\begin{enumerate}[resume]
	\item We ordered two \textbf{\textcolor{blue}{agua frescas}}. One was  \textbf{\textcolor{blue}{mango}} and the other was 
	\textbf{\textcolor{blue}{cantaloupe/watermelon}} on the waitress' recommendation. Wow! 
	\textbf{\textcolor{yellow}{both}} were so 
	\textbf{\textcolor{red}{fresh$_{positive}$}} and 
	\textbf{\textcolor{red}{delicious$_{positive}$}}.
	\\
	
	\textsc{ANAPHORA}(text=\textbf{\textcolor{yellow}{Both}})\\

	\textsc{COREFERENCE}(from=\textbf{\textcolor{blue}{Both}} to=\textbf{\textcolor{blue}{mango}}) \\
	\textsc{COREFERENCE}(from=\textbf{\textcolor{blue}{it}} to=\textbf{\textcolor{blue}{cantaloupe/watermelon}}) \\
	
\end{enumerate}

\end{document}